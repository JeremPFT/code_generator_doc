\documentclass[tikz]{standalone}
\usetikzlibrary{trees}

\usepackage{setspace}

\usepackage{comment}

\usepackage[hidelinks,
pdfauthor={Stefan Kottwitz},
pdftitle={Code Generator},
pdfsubject={no subject},
pdfkeywords={Mindmaps},
pdfproducer={Latex with hyperref},
pdfcreator={pdflatex},
pdfencoding=auto]{hyperref}

\usepackage{enumitem}
\setitemize{%
  noitemsep,
  topsep=0pt,
  parsep=0pt,
  partopsep=0pt,
  leftmargin=10pt}

% \tikzstyle{myNode} = [
% rectangle,
% rounded corners,
% draw=black,
% node distance=0.5cm]
% very thick,
% minimum width = 12em,
% minimum height=4.1em,
% text width=10em,

\newcommand{\content}[2]% #1=width #2=text
{\begin{minipage}{#1}
    \centering #2
  \end{minipage}%
}

\begin{document}

\begin{comment}
  :Title: A mindmap showing TeX projects supported by DANTE e.V

  :Tags: Mindmaps

  :Author: Stefan Kottwitz

  :Slug: servers

  A mindmap showing TeX projects supported by DANTE e.V., the german language
  TeX user group with home page at www.dante.de.

  DANTE sponsors their server costs for the web sites shown in the mindmap.

  - TeX forums
  - TeX galleries
  - TeX blogs
  - Tools, documentation and FAQ

  Less for explanation, as the code has been done quick and far from perfect,
  but is shown here to share.
\end{comment}

% level distance=10cm,
% sibling distance=3cm,
% node distance=10cm%
\begin{tikzpicture}%
  [%
  grow cyclic,
  every node/.style={%
    rectangle,
    rounded corners,
    draw = black,
    % top color = white,
    % bottom color = red!50,
    fill=red!50,
  },
  level 1/.append style={%
    level distance = 4cm,
    sibling angle = \directlua{tex.sprint(360 / 3)},
  },
  level 2/.append style={%
    level distance = 3cm,
    sibling angle = \directlua{tex.sprint(360 / 4)},
  },
  level 3/.append style={%
    level distance = 3cm,
    sibling angle = \directlua{tex.sprint(360 / 4)},
  }%
  ]
  \path
  %
  % central node: CODE GENERATOR
  %
  node[fill=black!40]{%
    \content{14em}{
      Code generator
      \begin{itemize}
      \item {\it automatize repetitive actions}
      \item {\it coding styles}
      \end{itemize}
    }
  }%
  %
  % INPUT: subprogram => define api / text => define syntax
  %
  child{node{\content{10em}{
        Input
        \begin{itemize}
        \item {\it using subprograms}
        \item {\it using text}
        \end{itemize}
      }}
    child[level distance = 5cm]{node{\content{6em}{
          Subprograms
          \begin{itemize}
          \item {\it define an API}
          \end{itemize}%
        }}}
    child{node{\content{12em}{
          Domain Specific Langage
          \begin{itemize}
          \item {\it define a parser}
          \end{itemize}%
        }}}
  }
  %
  % MODEL
  %
  child[level distance = 6cm]{node{\content{7em}{
        Model
        \begin{itemize}
        \item {\it data objects\\ definitions}
        \end{itemize}
      }}
    [level distance = 4cm, sibling angle = 45]
    child{node{\content{7ex}{
          Project%
        }}}
    child{node{\content{10ex}{
          UML elements%
        }}
      child{node[%
        rectangle,%
        % top color = white,
        % bottom color = black!30,
        fill=black!60,
        rounded corners = 0,
        ]{\content{10ex}{
            \it See UML specification}}}
    }
  }
  %
  % OUTPUT
  %
  child{
    node{
      \content{5em}{
        Output
      }%content
    }%node
    child{%
      node{%
        \content{4ex}{
          Ada%
        }%content
      }%node
    }%child
    child{%
      node{%
        \content{5em}{%
          Python
        }%content
      }%node
    }%child
  };

\end{tikzpicture}

\end{document}
